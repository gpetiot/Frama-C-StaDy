\documentclass[web]{frama-c-book}

\usepackage{graphicx,calc,wrapfig,tikz}

\newcommand{\framacversion}{\input{../../VERSION}}

\newcommand{\tool}[1]{\textsf{#1}\xspace}

\newcommand{\C}{\tool{C}}
\newcommand{\caml}{\tool{OCaml}}
\newcommand{\acsl}{\tool{ACSL}\index{ACSL@\tool{ACSL}}}
\newcommand{\gcc}{\tool{GCC}}
\newcommand{\Value}{\tool{Value}}
\newcommand{\framac}{\tool{Frama-C}}
\newcommand{\stady}{\tool{StaDy}}
\newcommand{\pathcrawler}{\tool{PathCrawler}}

\newenvironment{commands}%
{\par\hspace{2cm}\begin{tabular}{l}}%
{\end{tabular}}

\newcommand{\via}{\emph{via}\xspace}
\newcommand{\etc}{\emph{etc}\xspace}
\newcommand{\ie}{\emph{i.e.}\xspace}
\newcommand{\eg}{\emph{e.g.}\xspace}

% Index

\newcommand{\optionidx}[2]{\index{#2@\texttt{#1#2}}}
\newcommand{\codeidx}[1]{\index{#1@\texttt{#1}}}
\newcommand{\scodeidx}[2]{\index{#1@\texttt{#1}!#2@\texttt{#2}}}
\newcommand{\sscodeidx}[3]{%
  \index{#1@\texttt{#1}!#2@\texttt{#2}!#3@\texttt{#3}}}
\newcommand{\bfit}[1]{\textbf{\textit{\hyperpage{#1}}}}
\newcommand{\optionidxdef}[2]{\index{#2@\texttt{#1#2}|bfit}}
\newcommand{\codeidxdef}[1]{\index{#1@\texttt{#1}|bfit}}
\newcommand{\scodeidxdef}[2]{\index{#1@\texttt{#1}!#2@\texttt{#2}|bfit}}
\newcommand{\sscodeidxdef}[3]{%
  \index{#1@\texttt{#1}!#2@\texttt{#2}!#3@\texttt{#3}|bfit}}

\newcommand{\pragmadef}[1]{\texttt{#1}\index{Pragma!#1@\texttt{#1}}}
\newcommand{\optiondef}[2]{\texttt{#1#2}\optionidxdef{#1}{#2}}
\newcommand{\textttdef}[1]{\texttt{#1}\codeidxdef{#1}}

\newcommand{\optionuse}[2]{\texttt{#1#2}\optionidx{#1}{#2}}
\newcommand{\textttuse}[1]{\texttt{#1}\codeidx{#1}}

% Special environment
\definecolor{gris}{gray}{0.85}

\makeatletter
\newenvironment{important}%
{\hspace{5pt plus \linewidth minus \marginparsep}%
 \begin{lrbox}{\@tempboxa}%
   \begin{minipage}{\linewidth - 2\fboxsep}}
{\end{minipage}\end{lrbox}\colorbox{gris}{\usebox{\@tempboxa}}}
\makeatother

\newtheorem{convention}{Recommendation}[chapter]


\makeindex

\begin{document}

\coverpage{StaDy Manual}

\begin{titlepage}
\begin{flushleft}
\includegraphics[height=14mm]{cealistlogo.jpg}
\end{flushleft}
\vfill
\title{StaDy Manual}{Release \framacversion}
\author{Guillaume Petiot and Nikolay Kosmatov}
\begin{tabular}{l}
CEA LIST, Software Safety and Security Laboratory, Saclay, F-91191 \\
\end{tabular}
\vfill
\begin{flushleft}
  \textcopyright 2015 CEA LIST

This work has been supported by the project STANCE (FP7 317753).
\end{flushleft}
\end{titlepage}

\tableofcontents

%%%%%%%%%%%%%%%%%%%%%%%%%%%%%%%%%%%%%%%%%%%%%%%%%%%%%%%%%%%%%%%%%%%%%%%%%%%%%%%

%% \chapter*{Foreword}
%% \markright{}
%% \addcontentsline{toc}{chapter}{Foreword}

%% This is the user manual of \FramaC\footnote{\url{http://frama-c.com}}.  The
%% content of this document corresponds to the version \framacversion (\today) of
%% \FramaC. However the development of \FramaC is still ongoing: features
%% described here may still evolve in the future.

%% \section*{Acknowledgements}

%% We gratefully thank all the people who contributed to this document: Patrick
%% Baudin, Mickaël Delahaye, Philippe Hermann, Benjamin Monate and Dillon Pariente.

%%%%%%%%%%%%%%%%%%%%%%%%%%%%%%%%%%%%%%%%%%%%%%%%%%%%%%%%%%%%%%%%%%%%%%%%%%%%%%%

\chapter{Introduction}



Among
% the most powerful modern
formal verification techniques, \emph{deductive verification}
% can be used to prove
consists in establishing a rigorous mathematical proof that a given
% annotated
program meets its specification. When no confusion is possible, one also says
% for short
that deductive verification consists in ``proving a program''. It requires that
the program comes with a formal specification, usually given in special comments 
called \emph{annotations,} including function contracts (with
pre- and postconditions) and loop contracts (with loop variants and invariants).
The \emph{weakest precondition calculus} proposed by
Dijkstra~\cite{DBLP:books/ph/Dijkstra76} reduces any deductive verification
problem to %the one of 
establishing the validity of first-order
formulas called \emph{verification conditions}.
% is based on \emph{Weakest-Precondition calculus}
% initiated by Hoare, Floyd and Dijkstra in the late 1960's
%Hoare [Hoa69], Floyd [Flo67] and Dijkstra [Dij68]
% (see e.g. \cite{Hoare1969}) and


In modular deductive verification of a function $f$ calling another function
$g$, the roles of the pre- and postconditions of $f$ and of the callee $g$ are
dual.
% (see Fig.~\ref{fig:VerifFuncCall}).
The precondition of $f$ is assumed and its  postcondition must be proved, while
at  any call of $g$ in $f$, %\commentNK{ambiguous otherwise} 
the precondition of
$g$ must be proved before the call and its postcondition is assumed after the
call.
The situation for a function $f$ with one call to $g$ is presented in
Fig.~\ref{fig:verif-func-call}.
An arrow in this figure informally indicates that its initial point provides a
hypothesis for a proof of its final point.
% in Fig.~\ref{fig:verif-func-call} informally indicate basic implications to be
% proved for a function with one function call.
For instance, the precondition $\textit{Pre}_f$ of $f$ and the postcondition
$\textit{Post}_g$ of $g$ provide hypotheses for a proof of the postcondition
$\textit{Post}_f$ of $f$.
% formalizes this informal presentation).
The called function $g$ is proved separately.


%===========================================================
\input FigVerifFuncCall.tex
%===========================================================

To reflect the fact that some contracts become hypotheses
during deductive verification of $f$
we use the term \emph{subcontracts for $f$} 
to designate contracts of called functions and loops in $f$.


\textbf{Motivation.} 
One of the most important difficulties in deductive verification is the manual
processing of proof failures 
%that must be manually analyzed 
by the verification engineer since proof failures may have several
causes. 
Indeed, a failure to prove $\textit{Pre}_g$ in Fig.~\ref{fig:verif-func-call} may be due to
a \emph{non-compliance} of the code to the specification:
an error in the code \lstinline'code1', or a 
wrong specification $\textit{Pre}_f$ or $\textit{Pre}_g$ itself
that may incorrectly formalize the requirements.
The verification can also remain inconclusive because of 
a \emph{prover incapacity} to finish
a particular proof within an allocated time. 

In many cases, it is extremely difficult for the verification engineer 
to decide how to proceed: 
either suspect a non-compliance and look for an error in the code or check the specification, 
or suspect a prover incapacity, give up automatic proof and try to achieve an
interactive proof with a proof assistant (like \textsc{Coq} \cite{coq}).

A failure to prove the postcondition $\textit{Post}_f$ %in the situation shown in
(cf. Fig.~\ref{fig:verif-func-call}) is even more complex to analyze: along with a
prover incapacity or a non-compliance due to errors in the pieces of code
\lstinline'code1' and \lstinline'code2' or an incorrect
% due to an wrong
specification $\textit{Pre}_f$ or $\textit{Post}_f$, the failure can
also result from a \textit{too weak} postcondition $\textit{Post}_g$ of $g$, that
does not fully express the intended behavior of $g$. Notice that in this last case, the proof of $g$ can
still be successful. 
%However, 
The current automated tools for program proving
do not provide a precise 
% the verification engineer receives
indication on the reason of the proof failure.
%Some of them do not give any reason (like \framac/\Wp).
The most advanced tools (like \textsc{Dafny} \cite{Leino/FIDE14}) produce a
counter-example extracted from the underlying solver
that cannot precisely indicate if the verification engineer should look for 
a non-compliance,
or strengthen subcontracts (and which one of them), 
or consider adding additional lemmas or using interactive proof.
So the verification engineer must basically consider all possible reasons one
after another, maybe also trying a very costly interactive proof.
For a loop, %(cf. Fig.~\ref{fig:verif-loop})
the situation is similar,
and offers an additional challenge:
to prove the invariant preservation, whose failure 
can be due to several reasons as well.


The motivation of this work is twofold. First, we want to provide the
verification engineer with a more precise feedback indicating the reason of each
proof failure. Second, we look for a counter-example that either confirms the
non-compliance and demonstrates that the unproven predicate can indeed fail on a
test datum, or confirms a subcontract weakness
% hypothesis (the contract of a called function or a loop invariant)
showing on a test datum which subcontract is insufficient.
%Absence of counter-examples (at least for a reduced program domain)
%can indicate a prover incapacity to finish the unsuccessful proof.

\textbf{Approach and goals.}
%\commentAG{Ne pas mettre ce qu'on a fait (approach)
%et ce qu'on voudrait atteindre (goal) dans le m\^eme paragraphe.} 
%\commentNK{pas d'accord: comment on approche et pour quel objectif 
%peuvent aller ensemble}
The diagnosis of proof failures 
based on a counter-example generated by a prover can be imprecise
since from the prover's point of view,
the code of callees and loops in $f$ is replaced by the corresponding
subcontracts.
To make this diagnosis more precise, one should  
take into account their code as well as their contracts.
A recent study~\cite{Tschannen/14} proposed to use function inlining and loop unrolling.
We propose an alternative approach: to
use advanced test generation techniques in order to diagnose 
proof failures and produce counter-examples.
Their usage requires a translation of the annotated
%\commentAG{Ajouter une phrase
%expliquant la notion d'annotation de sp\'ecification en commentaire dans les
%langages de programmation usuels, ou ne jamais parler d'annotation.} 
%\commentNK{fait}
C program
into an executable C code suitable for testing.
%\commentAG{Cette notion de
%``suitable for testing'' est à mieux définir ou à reformuler.} 
%\commentNK{Not a term here, just English words}
Previous work
% \cite{Petiot/TAP14,Petiot/SCAM14}
addressed the generation of counter-examples by testing only for a
non-compli\-ance \cite{Petiot/TAP14} and proposed a rule-based formalization of
annotation translation
% into C code suitable to produce counter-examples by test generation
in that case~\cite{Petiot/SCAM14}.
The cases of subcontract weakness remained undetected and indistinguishable from
a prover incapacity.

The overall goal of the present work is to provide a methodology for a more
precise  identification of proof failure reasons in all these cases, to
implement it and to evaluate it in practice.
The proposed method is composed of two steps. The first step looks for
non-compliance. If no non-compliance is detected, the second step looks for a
subcontract weakness.
%\commentNK{Subcontract or Contract?} 
Another goal is to make this method automatic and suitable for a non-expert verification
engineer.
Following the modular verification approach,
we assume that the called functions have been verified before the caller $f$.
%, and the called functions respect their contracts.
To simplify the presentation, we also assume that 
the loops preserve their loop invariants, and focus
on other proof failures occurring during modular verification of $f$. 
(The proposed techniques can be adapted to the verification of a loop contract.)


%%%%%%%%%%%%%%%%%%%%%%%%%%%%%%%%%%%%%%%%%%%%%%%%%%%%%%%%%%%%%%%%%%%%%%%%%%%%%%%

\chapter{Functioning and user interface}

The StaDy prototype has two main functionalities : non-compliance detection and subcontract weakness detection.

\section{Non-compliance detection}

StaDy tries to exhibit a test case whose execution provokes an annotation violation. Non-compliance detection is the default mode of StaDy. One can detect non-compliances between the code and its specification in file F, starting with function M, using the command :

\begin{shell}
frama-c F -main M -stady
\end{shell}

In the user interface of Frama-C, an annotation violation is pictured with a red bullet, as we can see in Fig.~\ref{fig:gui}

\begin{figure}\centering
  \includegraphics[scale=.4]{ppt_invalid.png}
  \caption{Invalid annotation in the Graphical User Interface of Frama-C
    \label{fig:gui}}
\end{figure}

\section{Subcontract weakness detection}

StaDy tries to exhibit a test case whose execution does not provoke an annotation violation but that is not proved by deductive verification because of a too weak subcontract (loop invariant or called function contract). The subcontract weakness detection in file F starting with function M is done with :

\begin{shell}
  frama-c F -main M -stady -stady-swd C
\end{shell}

where C is a comma-separated list of subcontract identifiers. The subcontract identifiers are the statement identifiers of the corresponding loops (for loop contracts) and function calls (for function contracts).

For now, there is no strategy or heuristic implemented in StaDy to automatically submit a set of subcontract identifiers

%%%%%%%%%%%%%%%%%%%%%%%%%%%%%%%%%%%%%%%%%%%%%%%%%%%%%%%%%%%%%%%%%%%%%%%%%%%%%%%


\appendix


%%%%%%%%%%%%%%%%%%%%%%%%%%%%%%%%%%%%%%%%%%%%%%%%%%%%%%%%%%%%%%%%%%%%%%%%%%%%%%%

\cleardoublepage
\phantomsection
\addcontentsline{toc}{chapter}{\bibname}
\bibliographystyle{plain}
\bibliography{./manual}

%%%%%%%%%%%%%%%%%%%%%%%%%%%%%%%%%%%%%%%%%%%%%%%%%%%%%%%%%%%%%%%%%%%%%%%%%%%%%%%

\cleardoublepage
\phantomsection
\addcontentsline{toc}{chapter}{\listfigurename}
\listoffigures

%%%%%%%%%%%%%%%%%%%%%%%%%%%%%%%%%%%%%%%%%%%%%%%%%%%%%%%%%%%%%%%%%%%%%%%%%%%%%%%

\cleardoublepage
\phantomsection
\addcontentsline{toc}{chapter}{\indexname}
\printindex

\end{document}
